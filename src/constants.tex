% This is where you'd set up your primary set of constants that will be used
%   in headers, and generally throughout the document.
\def\EmailAddress{name@domain.tld}
\def\PhoneNumber{555-555-5555}
\def\Name{Firstname Lastname}
\def\Address{\mbox{12 -- 4215 Somewhere St.} \mbox{City, Province, A1B 2C3}}

\def\GitHubUsername{github}
\def\LinkedInUsername{linkedin}

% Left and right lower headers for below the mean header ing
\newcommand{\LeftSubheader}{%
    \href{https://github.com/\GitHubUsername}%
        {\raisebox{-.25\height}{\GitHubHeader} GitHub/\GitHubUsername}%
}
\newcommand{\RightSubheader}{%
    \href{https://www.linkedin.com/in/\LinkedInUsername}%
        {\raisebox{-.25\height}{\LinkedInHeader} LinkedIn/\LinkedInUsername}%
}

% Parameters that can be tuned to manage padding when resumes need a little
%   more adjustments to get the right look and feel.
% Also can be used to tip the scales between one-pagers vs. multi-page resumes.
\def\DetailParSkip{1pt}
\def\SubheaderPadding{60pt}

% Header with address only
\newcommand{\CenterSubheader}{\Address}

% Use this to control how much content appears on the final document. Both your
%   resume, and curriculum vitae can be built in the same document, and can be
%   selected for building using this conditional.
\newif\ifcv
\cvfalse
